%%=============================================================================
% Создавайте лекции, используя этот шаблон. Не забывайте удалять из шаблона
% лишние комментарии.
% Шаблон начинается с указания класса lecture-notes.
%%=============================================================================
\documentclass[russian]{lecture-notes}

% В классе lecture-notes уже подключен пакет babel для русского языка и указана
% кодировка utf8, здесь это не нужно делать заново

% Подключаем дополнительные пакеты. Сюда добавляются все необходимые пакеты,
% не только те, которые указаны ниже
\usepackage[final]{graphicx}  % возможность вставлять изображения в текст
\usepackage{amsmath} % дополнительные возможности для набора математики
\usepackage{svg} % возможность вставлять svg изображения. Этот пакет работает
                 % только после специальной настройки, о ней написано в
                 % инструкции к пакету lecture-notes
\usepackage{subcaption} % пакет нужен, чтобы можно было ставить несколько рисунктов рядом
\usepackage{timestamps} % возможность указывать моменты времени на youtube видео

% Задаем заголовок конспескта

% сначала название лекции, если это конспект отдельной лекции. Или курса, если
% это конспект одной из лекций курса.
\title{Курс лекций об основах математики}
% в команде \subtitle название отдельной лекции. Если это не нужно указывать,
% то надо удалить команду \subtitle
\subtitle{Лекция 1}
% указываем автора лекций. Несколько человек разделяются командой \and
\lecturer{Дональд Кнут}
% указываем авторов конспекта. Несколько человек разделяются командой \and
\notesauthor{С. Тудент \and С. Тудентка \and А. Спирант}
% дата лекции. Можно не указывать, в этом случае нужно полностью удалить команду \date
\date{1 января 1970}

% если подклюючен пакет timestamps, указываем идентификатор youtube видео, его
% можно взять из строки адреса с видео.
% Видео для примера выбрано случайным образом, оно никак не связано с текстом
% в этом файле.
\youtubevideo{zIYfYT5THDY}

% начало документа
\begin{document}
% сначала формируем заголовок, он составляется из всей указанной выше информации
\maketitle

% ставим отметку времени в формате минут:секунд.
\timestamp{00:00}
% название раздела
\section{Примеры использования возможностей \LaTeX}
\subsection{Формулы}

Пишем текст конспекта.
Если нужно вставить формулу, пользуемся долларами для формул внутри абзаца,
например, $x_{1,2}=\frac{-b\pm\sqrt{b^2-4ac}}{2a}$.
Если нужна вынесенная формула, оформляем ее внутри
\texttt{\textbackslash[} и \texttt{\textbackslash]}:
\[x_{1,2}=\frac{-b\pm\sqrt{b^2-4ac}}{2a}\]

% подраздел, в нем тоже указываем отметку времени
\timestamp{00:45}
\subsection{Перечисления}

Сначала пример перечисления без нумерации.

\begin{itemize}
    \item Пример перечисления.
    \item Второй пункт.
    \item Третий пункт.
\end{itemize}
Теперь пример нумерованного списка.
\begin{enumerate}
    \item Первый пункт;
    \item Второй пункт;
    \item Третий пункт.
\end{enumerate}

\timestamp{09:23}
\subsection{Рисунки и ссылки}
Если мы захотим сослаться на этот подраздел, надо поставить метку в этом месте.
Название метки может быть произвольным, здесь я использую префикс sec:, чтобы
было понятно, что это ссылка на раздел, но такое именование не обязательно.
\label{sec:images-and-refs}

И сразу можно сослаться на раздел, например, сейчас вы читаете раздел~\ref{sec:images-and-refs}.
Помните, что при использовании ссылок может потребоваться компилировать файл дважды.
Дело в том, что \TeX\ пользуется информацией о положении меток, полученных при предыдущей компиляции.
Поэтому, если скомпилировать один раз, в документе могут получиться неправильные ссылки.

Все рисунки оформляются в окружении figure. Это окружение позволяет рисунку перемещаться,
и \TeX\ выберет для него оптимальное положение на странице. Вы не должны особенно заботиться о положении
рисунков, не страшно, если \TeX\ поставил их не туда, куда бы вам хотелось, очень редко нужно тратить
усилия на то, чтобы заствить рисунок встать именно туда, куда хочется.
В квадратных скобках пишутся пожелания о положении рисунка. h — здесь, t — сверху страницы, b — снизу. p — на
отдельной странице для рисунков. Если поставить !, то \TeX\ чуть больше постарается, чтобы поставить рисунок
именно туда, куда написано. Например, пишите [h!], чтобы попросить рисунок остаться именно там, где
он указан.
Но, какие бы пожелания ни были, \TeX\ может их не выполнить, если у него не получится расположить
изображение так, как ему кажется правильно.

\begin{figure}[htb]
    \centering % центруем рисунок
    % команда на вставку рисунка. Указываем ширину, в данном случае она указывается как 40% от ширины
    % текста.
    % указывать размер можно по-разному, например, вот так в абсолютных величинах: [width=3cm, height=4cm]
    % или можно вообще не указывать размер, тогда квадратные скобки и текст внутри не пишутся.
    \includegraphics[width=0.4\textwidth]{xkcd1301.png}
    % подпись рисунка
    \caption{Достоверность информации в зависимости от расширения файла}
    % метка, чтобы потом можно было сослаться на рисунок. Префикс fig:
    \label{fig:example1}
\end{figure}

\subsection{SVG рисунок}

Как заставить работать SVG изображения написано в инструкции по установке класса
lecture-notes.
Здесь отличие от вставки обычных изображений заключается только в том, что
необходимо заменить команду команду \textbackslash includegraphics на
\textbackslash includesvg:

\begin{figure}[htb]
    \centering
    \includesvg[width=0.6\textwidth]{image-from-inkscape-1.svg}
    \caption{Вставка векторного изображения}
    \label{fig:svg-example}
\end{figure}


\subsubsection{Два рисунка рядом}

Далее пример двух изображений рядом на одном рисунке.
Чтобы это работало, в начале файла нужно подключить пакет subcaption.

На рисунке~\ref{fig:example-2} расположены два изображения, одно слева, на него можно сослаться
как на рисунок~\ref{fig:example-2a}, и одно справа: рисунок~\ref{fig:example-2b}.

\begin{figure}[h!]
    \centering
    % Первый подрисунок указываем размер, буква [t] означает, что картинка прижмется к верху
    \begin{subfigure}[t]{70mm}
        %Здесь \textwidth это ширина всего пространства для подрисунка
        \includegraphics[width=\textwidth]{xkcd1301.png}
        \subcaption{Левый рисунок}
        \label{fig:example-2a}
    \end{subfigure}
    \quad %это широкий пробел между двумя рисунками
    \begin{subfigure}[t]{50mm}
        \includegraphics[width=\textwidth]{xkcd1301.png}
        \subcaption{Правый рисунок}
        \label{fig:example-2b}
    \end{subfigure}
    \caption{Пример двух изображений рядом на одном рисунке}
    \label{fig:example-2}
\end{figure}


\subsection{Таблицы}

Пример таблицы. [h!] означает, что таблицу нужно попытаться поставить именно здесь. Окружение
table аналогично окружению figure, оно позаботится о том, чтобы хорошо расположить таблицу.
Окружение tabular задает таблицу, у него много возможностей, о них можно читать в учебниках.
\begin{table}[h!]
    \centering
    % Заголовок таблицы, в отличие от заголовка рисунка, пишется перед таблицей
    \caption{Пример таблицы}
    \label{tab:example1}
    \begin{tabular}{|p{5cm}|c|}
        \hline
        \centering
        Вопрос & ? \\
        \hline
        Ответ & 42 \\
        \hline
    \end{tabular}
\end{table}

\end{document}
