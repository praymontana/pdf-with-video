%%=============================================================================
% Создавайте лекции, используя этот шаблон. Не забывайте удалять из шаблона
% лишние комментарии.
% Шаблон начинается с указания класса lecture-notes.
%%=============================================================================
\documentclass[russian]{lecture-notes}

\usepackage[final]{graphicx}
\usepackage{amsmath}
\usepackage{svg}
\usepackage{timestamps}

\title{Рекуррентные тропические последовательности}
\lecturer{Дональд Кнут}
\notesauthor{С. Тудент \and С. Тудентка \and А. С. Пирант}
\date{вчера}
\subtitle{Лекция 2. Тоже о чем-то}

\begin{document}
\youtubevideo{zIYfYT5THDY}

\maketitle
\timestamp{00:00}
\section{Классические рекуррентные последовательности}

 Сначала мы рассмотрим классические рекуррентные последовательности, для которых уже есть завершённая теория, а затем перейдём к тропическим.
\timestamp{00:45}
\subsection{Что такое классическая реккурентная последовательность?}
\end{document}
