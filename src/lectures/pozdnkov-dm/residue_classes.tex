\documentclass[russian]{lecture-notes}

\usepackage[final]{graphicx}
\usepackage{subcaption}
\usepackage{timestamps}
\usepackage{hyperref}
\usepackage{float}
\usepackage{amsmath}
\usepackage{amssymb}
\usepackage{tikz}

\renewcommand{\arraystretch}{1.3}
\newcommand{\frc}[2]{\raisebox{2pt}{$#1$}\big/\raisebox{-3pt}{$#2$}}
\newcommand{\divs}{\mathrel{\raisebox{-2pt}{\vdots}}}
\newcommand{\klas}[1]{\overline{#1}}

\title{Лекция <<Классы вычетов. Теорема Ферма. Китайская теорема об остатках>>}
\lecturer{Поздняков Сергей Николаевич}
\notesauthor{Кацер Евгений}
\date{21 марта 2018 г.}
\youtubevideo{DGPUOBR8ayU}

\begin{document}
	\maketitle

\begin{problem}[С прошлой лекции]
	\timestamp{09:00}
	Определить признак делимости на 7 числа, заданного в двоичной системе счисления. Признак представить как в операциях десятичной, так и двоичной систем счисления. Рассмотрим оба варианта:
	\begin{itemize}
		\item Наше число, заданное в двоичной системе счисления, выглядит так: 
			\[x=b_n2^n + b_{n-1}2^{n-1} + \dots + b_42^4+ b_32^3 + b_22^2+b_12^1+b_0\]	
			Свойство делиться на $7$ сохранится, если мы уменьшим число на величину, кратную семи, поэтому в этом выражении можно заменить степени двойки их остатками от деления на $7$. Заметим, что остаток от деления $2^3 = 8$ на $7$ равен $1$. Тогда остаток от вдвое большего числа $2^4$ на $7$ можно найти, умножив предыдущий остаток на $2$, аналогично, остаток от деления $2^5$ получается умножением предыдущего остатка на $2$, что дает $4$, дальнейшее умножение на $2$ дает $8$, но, как уже говорилось, можно вместо $8$ взять его остаток от деления на $7$, и мы снова получим $1$, то есть последовательность остатков от деления степеней $2$ на $7$ зациклится (в дальнейшем будем использовать символ $\equiv$, означающий, что у чисел одинаковые остатки)
			\[x=b_n2^n + b_{n-1}2^{n-1} + \underbrace{\dots + b_4\underset{\equiv 2}{2^4}+ b_3\underset{\equiv 1}{2^3}} + \underbrace{b_2\underset{\equiv 4}{2^2}+b_1\underset{\equiv 2}{2}+b_0\underset{\equiv 1}{1}}_{\text{цикл}}\]
			Тогда, если 
			\[
				b_0 + 2b_1 + 4b_2 + b_3 + 2b_4 + 4b_5 + \ldots \ \divs 7 \implies x \divs 7 \ (\ast) 
			\]~--- признак делимости на 7 для операций, выполняемых в десятичной системе счисления.
		\item В двоичной системе полученные на предыдущем шаге коэффициенты $1, 2, 4$~--- степени двойки, следовательно, если рассматривать $(\ast)$ в двоичной системе счисления, коэффициенты $4b_2 + 2b_1 + b_0$ можно записать как $\klas{b_2b_1b_0}$, $4b_5 + 2b_4 + b_3$ можно записать $\klas{b_5b_4b_3}$ и так далее. Получается, что число можно разбить на тройки (если чисел до тройки не хватает, дополнить впереди нулями). А потом сложить эти двоичные числа, и, если сумма будет делиться на $7$, то и $x$ будет делиться на $7$: 
		\[ 
			\klas{b_0b_1b_2} + \klas{b_3b_4b_5} + \ldots \divs 7 \implies x \divs 7 
		\]
		~--- признак делимости на 7 в двоичной системе счисления.
	\end{itemize}
	
	\begin{example}
		Проверим по обоим признакам $35_{10}=100011_2$:
		\begin{itemize}
			\item Проверим по десятичному признаку: для этого добавим веса и запишем сумму:
			\[\underset{4}{1}\underset{2}{0}\underset{1}{0}\underset{4}{0}\underset{2}{1}\underset{1}{1} \implies 4\cdot1+2\cdot1+1\cdot1=7 \divs 7\]
			\item Проверим по двоичному признаку, для этого разобьем число на тройки и сложим их вместе:
			\[\overbrace{100}\overbrace{011} \implies 100+011=111=7 \divs 7\]
		\end{itemize}
	\end{example}
\end{problem}

\begin{note}
	При проверке делимости на $7$ в двоичной системе счисления не обязательно переводить результат в десятичную систему счисления, так как при многократном применении признака у нас получится число, состоящее из трех двоичных цифр, делящееся на $7$. Таких чисел всего два: $000$ (ноль) и $111$ (семь).
\end{note}

\section{Классы вычетов}
\timestamp{16:03}

\begin{example}
	Если к нечетному числу прибавить нечетное~--- получится четное. Если нечетное число умножить на четное~--- получится четное. Мы можем оперировать этими понятиями. Мы не складывали конкретные числа $3$ и $7$, мы работали с  классами чисел, то есть мы сразу сложили все нечетные числа со всеми нечетными и заявили, что  полученное множество совпадает со множеством четных чисел или является его частью. То, с чем мы сейчас работаем, называется \emph{классами вычетов}. Их можно привязать к остаткам естественным образом, например:
	\begin{itemize}
		\item\label{exp:chet}Все нечетные числа можно обозначить как множество чисел, дающих остаток 1 при делении на 2. Обозначают палочкой над числом $\klas{1}$ и называют классом вычетов.
		\[ \{ 2k+1 \mid k \in \mathbb{Z}\} = \klas{1} \]
		\item\label{exp:nechet}Все четные числа можно обозначить, как множество чисел, дающих остаток 0 при делении на 2.
		\[ \{ 2k \mid k \in \mathbb{Z}\} = \klas{0}\]
	\end{itemize}
\end{example}

Классы вычетов обозначают $\mathbb{Z}_2$ или $\frc{\mathbb{Z}}{2\mathbb{Z}}$. Последнее обозначение связано с понятием факторизации. То, что стоит под чертой, показывает, что все четные числа рассматриваются, как один объект, и тогда множество всех целых чисел относительно $2\mathbb{Z}$ распадается на две части. Одни попадают в $2\mathbb{Z}$ (четные числа), а другие~--- нет (нечетные числа). Значок деления выбран не случайно, ведь когда мы записываем дробь $\frc{6}{3}$, мы можем трактовать это как действие <<разделить множество из $6$ объектов на части по $3$ объекта в каждой>>.

\begin{table}[H]
	\caption{Арифметика в $\mathbb{Z}_2 = \frc{\mathbb{Z}}{2\mathbb{Z}}$}
	\centering
	\label{tab:arif}
	\begin{subtable}[H]{0.49\linewidth}
		\centering
		\subcaption{Арифметика сложения}
		\label{tab:arif1}
		\begin{tabular}{|c|c|c|}
			\hline
			$+$ & $\klas{0}$ & $\klas{1}$ \\ \hline
			$\klas{0}$ & $\klas{0}$ & $\klas{1}$ \\ \hline
			$\klas{1}$ & $\klas{1}$ & $\klas{0}$ \\ 
			\hline
		\end{tabular}
	\end{subtable}
	\hfill
	\begin{subtable}[H]{0.49\linewidth}
		\centering
		\subcaption{Арифметика умножения}
		\label{tab:arif2}
		\begin{tabular}{|c|c|c|}
			\hline
			$\cdot$ & $\klas{0}$ & $\klas{1}$ \\ \hline
			$\klas{0}$ & $\klas{0}$ & $\klas{0}$ \\ \hline
			$\klas{1}$ & $\klas{0}$ & $\klas{1}$ \\ 
			\hline
		\end{tabular}
	\end{subtable}
\end{table}

\begin{enumerate}
	\item 	\begin{enumerate}
				\item Четное$+$Четное$=$Четное (как и $0 + 0 = 0$) \\
					Четное$+$Нечетное$=$Нечетное (как и $0 + 1 = 1$) \\
					Нечетное$+$Нечетное$=$Четное (как и $1 + 1 = 0$)
				\item Четное$*$Четное$=$Четное (как и $0 \cdot 0 = 0$) \\
				Четное$*$Нечетное$=$Четное (как и $0 \cdot 1 = 0$) \\
				Нечетное$*$Нечетное$=$Нечетное (как и $1 \cdot 1 = 1$)
			\end{enumerate}
\end{enumerate}

\begin{note}
	С помощью такой арифметики мы можем осуществлять арифметические операции сразу над бесконечными количествами чисел.
\end{note}

\begin{example}
	\timestamp{20:05}
	Рассмотрим $\mathbb{Z}_5$ и $\mathbb{Z}_6$.
	\begin{table}[H]
		\caption{Арифметика в $\mathbb{Z}_5$ и $\mathbb{Z}_6$}
		\centering
		\label{tab:arif}
		\begin{subtable}[H]{0.49\linewidth}
			\centering
			\subcaption{Умножение в $\mathbb{Z}_5$}
			\label{tab:arif1}
			\begin{tabular}{|c|c|c|c|c|c|}
				\hline
				$\cdot$ & $\klas{0}$ & $\klas{1}$ & $\klas{2}$ & $\klas{3}$ & $\klas{4}$\\ \hline
				
				$\klas{0}$  & $\klas{0}$ & $\klas{0}$ & $\klas{0}$ & $\klas{0}$ & $\klas{0}$\\ \hline
				
				$\klas{1}$ & $\klas{0}$ & $\klas{1}$ & $\klas{2}$ & $\klas{3}$ & $\klas{4}$\\ \hline
				
				$\klas{2}$ & $\klas{0}$ & $\klas{2}$ & $\klas{4}$ & $\klas{1}$ & $\klas{3}$\\ \hline
				
				$\klas{3}$ & $\klas{0}$ & $\klas{3}$ & $\klas{1}$ & $\klas{4}$ & $\klas{2}$\\ \hline
				
				$\klas{4}$ & $\klas{0}$ & $\klas{4}$ & $\klas{3}$ & $\klas{2}$ & $\klas{1}$\\ \hline
			\end{tabular}
		\end{subtable}
		\hfill
		\begin{subtable}[H]{0.49\linewidth}
			\centering
			\subcaption{Умножение в $\mathbb{Z}_6$}
			\label{tab:arif2}
			\begin{tabular}{|c|c|c|c|c|c|c|}
				\hline
				\centering
				\label{tab:arif2}
				$\cdot$ & $\klas{0}$ & $\klas{1}$ & $\klas{2}$ & $\klas{3}$ & $\klas{4}$ & $\klas{5}$\\ \hline
				
				$\klas{0}$  & $\klas{0}$ & $\klas{0}$ & $\klas{0}$ & $\klas{0}$ & $\klas{0}$ & $\klas{0}$\\ \hline
				
				$\klas{1}$ & $\klas{0}$ & $\klas{1}$ & $\klas{2}$ & $\klas{3}$ & $\klas{4}$ & $\klas{5}$\\ \hline
				
				$\klas{2}$ & $\klas{0}$ & $\klas{2}$ & $\klas{4}$ & $\klas{0}$ & $\klas{2}$ & $\klas{4}$\\ \hline
				
				$\klas{3}$ & $\klas{0}$ & $\klas{3}$ & $\klas{0}$ & $\klas{3}$ & $\klas{0}$ & $\klas{3}$\\ \hline
				
				$\klas{4}$ & $\klas{0}$ & $\klas{4}$ & $\klas{2}$ & $\klas{0}$ & $\klas{4}$ & $\klas{2}$\\ \hline
				
				$\klas{5}$ & $\klas{0}$ & $\klas{5}$ & $\klas{4}$ & $\klas{3}$ & $\klas{2}$ & $\klas{1}$\\ \hline
			\end{tabular}
		\end{subtable}
	\end{table}
\end{example}

\begin{note}
	Заметим, что классы вычетов $\klas{2}$ и $\klas{3}$ в $\mathbb{Z}_6$ являются делителями нуля. Это означает, что найдутся такие ненулевые классы, которые при умножении на них дадут класс вычетов 0.  Например, для $\klas{3}$ это классы $\klas{2}$ и $\klas{4}$.
	
	<<Хорошие>> строчки или столбцы получаются, если число и модуль класса вычета взаимно просты. В этих строках и столбцах встречаются все варианты классов вычетов.
	
	Структура, образуемая $\mathbb{Z}_5$, называется полем (определены не только сложение, вычитание и умножение, но и деление на любой класс, кроме $\klas{0}$). Например, $\klas{4} / \klas{3} = \klas{3}$ или $\klas{4} / \klas{2} = \klas{2}$. Мы можем это сделать, так как в каждом столбце или строчке есть все числа. Полями также являются множества $\mathbb{Q}$, $\mathbb{R}$, $\mathbb{C}$ и др.

	Структура, образуемая $\mathbb{Z}_6$ называется~--- кольцо (все операции, кроме деления определены). Например, мы не можем разделить $\klas{4}$ на $\klas{3}$. Кольцами также являются многочлены (т.к. всегда поделить многочлен на многочлен не получится), множество $\mathbb{Z}$ (т.к. тоже всегда поделить целое число на целое не получится) и др.
	
	Поле вычетов по простому модулю всегда конечное, и в нем справедливы все операции, свойства и формулы, выведенные в математике.
	
	Когда мы работаем с классами вычетов, мы можем записать множество нечетных чисел~[\ref{exp:chet}], как:
	\[
		\{ 2k+1 \mid k \in \mathbb{Z}\} = \{ 2k+3 \mid k \in \mathbb{Z}\}
	\]
	$\mathbb{Z}$~--- бесконечное, поэтому не важно, какое число мы возьмем в качестве базового (от которого будем вести отчет). Таким образом в нашей арифметике мы можем не ограничивать себя выбором конкретного представителя класса вычетов, а выбирать тот, который в данной ситуации более удобен:
	\[
		\klas{1} = \klas{3} = (\klas{-1})
	\]
	
	С классами вычетов мы можем работать, как с целыми числами. Иногда при простых вычислениях это очень удобно.
\end{note}

Рассмотрим эффект, возникающий при работе в конечных полях:
\begin{example}
	Решить уравнение
	$x^2+\klas{2}x+\klas{3}=\klas{0}$ в $\mathbb{Z}_{17}$. 
	
	Все операции и свойства, которые работают с вещественными, рациональными числами, переносятся на конечные поля. Если классы вычетов образуют поле, то для работы с ними можно использовать выведенные в школе формулы. Например, проверим применимость к нашей задаче известной формулы квадратного уравнения:
	\[ x_{1,2} = -\klas{1} \pm \sqrt{\klas{1}-\klas{3}} = -\klas{1} \pm \sqrt{-\klas{2}} = -\klas{1} \pm \sqrt{-\klas{2}+\klas{17}} =\]
	\[ = -\klas{1} \pm \sqrt{\klas{15}} = -\klas{1} \pm \sqrt{\klas{32}} = -\klas{1} \pm \sqrt{\klas{49}} = -\klas{1} \pm \klas{7} \]
	
	$x_1 = \klas{6}$
		
	$x_1 = -\klas{8} = \klas{9}$
	
	Проверка:
	\begin{itemize}
		\item $6^2 + 2 \cdot 6 + 3 = 36 + 12 + 3 = 51 \equiv 0 \pmod{17}$
		\item $9^2 + 2 \cdot 9 + 3 = 81 + 18 + 3 = 102 \equiv 0 \pmod{17}$
	\end{itemize}
	\label{exa:first}
\end{example}

\begin{note*}
	В рассмотренном примере~[\ref{exa:first}] всего 17 классов вычетов, поэтому корни можно найти перебором. В этом заключается достоинство арифметики остатков. Не всегда нужно пользоваться формулами, иногда можно использовать подбор.
\end{note*}

\begin{note*}
	$\mathbb{Z}_m$ содержит $m$ классов вычетов: $\klas{0}, \ \klas{1}, \ \dots \ , \  \klas{m-1}$~--- \emph{полная система классов вычетов} ($\klas{a}$~--- представитель класса вычетов).
\end{note*}

\begin{problem}
	\timestamp{39:07}
	\mbox{}
	\begin{enumerate}
		\item Доказать, что если добавить ко всем представителям класса вычетов минусы, они все равно будут образовывать полную систему классов вычетов:
		\[ \klas{0}, \ \klas{-1}, \ \dots \ , \  \klas{1-m} \]
		\item Доказать, что если добавить ко всем представителям класса вычетов $a$, они все равно будут образовывать полную систему классов вычетов:
		\[ \klas{0+a}, \ \klas{1+a}, \ \dots \ , \  \klas{m-1+a} \]
		\item Найти такие $a$, при которых умножение на $a$ тоже даст полную систему классов вычетов:
		\[ \klas{0}, \ \klas{a}, \ \dots \ , \  \klas{(m-1)*a} \]
	\end{enumerate}
\end{problem}

\begin{note}
	Когда мы говорим о классах вычетов, мы имеем в виду множество целых чисел. Иногда удобно использовать свойства, присущие целым числам, делимость, а иногда арифметику классов вычетов и не думать ни о каких целых числах, как в примере~[\ref{exa:first}]. 
\end{note}

А теперь попробуем наоборот свести свойства классов вычетов к свойствам целых чисел. Для этого введем определение:
\begin{definition}
	\timestamp{42:02}
	$a \equiv b \pmod m$ ($a$ сравнимо с $b$ по модулю $m$), если $\exists \ q_1,q_2 : 
	\begin{cases}
		a = m \cdot q_1 + r \\
		b = m \cdot q_2 + r
	\end{cases}$
	($a$ и $b$ дают одинаковые остатки от деления на $m$). 
\end{definition}

\begin{theorem}
	$a \equiv b \pmod m \iff (a-b) \divs{m}$
\end{theorem}

\begin{proof}
	$\Rightarrow$ По определению сравнимости $a$ можно записать так: $a = m \cdot q_1 + r$, а $b$ так: $b = m \cdot q_2 + r$, тогда:
	\[
		a - b = m \cdot q_1 + r - (m \cdot q_2 + r) = m \cdot q_1 - m \cdot q_2 = m \cdot (q_1 - q_2) 
	\]
	Обозначим $(q_1 - q_2)$ за $q$, тогда:
	\[
		a - b = m \cdot q
	\]
	У нас получилось в точности определение делимости.
	
	\noindent$\Leftarrow$ доказать самостоятельно.
	
\end{proof}

\subsection{Свойства сравнений:}
\label{sec:prop}
\begin{enumerate}
	\item 	$\begin{cases}
				a \equiv b \pmod m \\
				b \equiv c \pmod m
			\end{cases} \implies a \equiv c \pmod m$.
			\label{sravn:1}
	\begin{proof}
		Числа $a$ и $b$ дают одинаковые остатки при делении на $m$. Обозначим его $r$. Числа $b$ и $c$ тоже дают одинаковые остатки при делении на $m$, но остаток от деления $b$ на $m$ мы уже обозначили $r$, значит остаток деления $c$ на $m$ тоже $r \implies a \equiv c \pmod m$. Также это свойство можно доказать с помощью свойств делимости.
		
	\end{proof}
	\item 	\begin{enumerate}
				\item $a \cdot k \equiv b \cdot k \pmod{m \cdot k} \iff a \equiv b \pmod m$.
				\label{sravn:2}
				\begin{proof}		
					По теореме (о сведении сравнения к делимости): 
					\[
						a \cdot k \equiv b \cdot k \pmod{m \cdot k} \iff a \cdot k - b \cdot k \divs m \cdot k
					\]
					Разделим на $k$, чтобы деление было определено в условии надо дописать, что $k \ne 0$. После сокращения получим:
					\[
						a - b \divs m
					\]
					а это по теореме то же, что и:
					\[
						a \equiv b \pmod m
					\]
				\end{proof}
				Свойство после добавления условия: 
				\[
					a \cdot k \equiv b \cdot k \pmod{m \cdot k} \text{ и } k \ne 0 \iff a \equiv b \pmod m
				\]
				\item $a \cdot k \equiv b \cdot k \pmod m \iff a \equiv b \pmod m$. Здесь мы не умножаем модуль на $k$. Возможно ли, что и (a), и (b) справедливы одновременно? Давайте проведем доказательство и выясним.
				\label{sravn:3}
				\begin{proof}
					$\Rightarrow$ По теореме (о сведении сравнения к делимости): 
					\[
						a \cdot k \equiv b \cdot k \pmod m \iff (a - b) \cdot k \divs m 
					\]
					У нас получилось, что произведение разности и $k$ делится на $m$. Нужно доказать, что разность делится на $m$. Это гарантировано только в одном случае, если $k$ взаимно просто с $m$. При выполнении этого условия:
					\[
						(a - b) \cdot k \divs m \implies a-b \divs m
					\]
					а это по теореме то же, что и:
					\[
						a \equiv b \pmod m
					\]
					$\Leftarrow$ Если $a-b \divs m$, то $(a - b) \cdot k \divs m$ всегда.
					\begin{remark}
						Заметим, что для доказательства второго свойства в одну сторону нужно ограничение на $k$, а в другую — нет.
					\end{remark}
				\end{proof}
				Свойство после добавления условия:
				\[
					a \cdot k \equiv b \cdot k \pmod m \text{ и } k \text{~--- вз. просто с } m \implies a \equiv b \pmod m
				\]
				\[
					a \equiv b \pmod m \implies a \cdot k \equiv b \cdot k \pmod m
				\]
			\end{enumerate}
\end{enumerate}

\begin{example}
	$36x \equiv 24y \pmod{64}$ разделим на $4$
	
	$9x \equiv 6y\divs 3 \pmod{16}$ (3 взаимно просто с 16) $\implies
	3x \equiv 2y \pmod{16}$
\end{example}

\begin{problem*}
	\timestamp{54:40}
	Доказать, что уравнение $x^2 - y^2 = 10^n$ не имеет решений при $n=1$
	
	\[ x^2 - y^2 = (x-y) \cdot (x+y) = 10 \]
	
	Произведение этих множителей~--- четное число $\implies$ по крайней мере один из множителей должен быть четным. Но если $x-y$~--- четный, то и $x+y$~--- четный, и наоборот, но 10 не раскладывается на два четных множителя.
	
	Переформулируем на язык сравнений: если $x-y \equiv 0 \pmod 2 \implies x+y \equiv 0 \pmod 2)$ 
	
	\[ x-y=x+(-1) \cdot y, \] а $(-1)=1$ по модулю 2 $\implies (x-y) \equiv (x+y) \equiv 0 \pmod 2$
	\label{zad:uravn} 
\end{problem*}

\begin{problem*}
	Вывести новые свойства сравнений, чтобы перевести рассуждения из задачи~\ref{zad:uravn} полностью на язык сравнений и доказать:
	\[
		x^2 \equiv y^2 \pmod 2 \implies x^2 \equiv y^2 \pmod 4
	\]
\end{problem*}

\begin{center}
	\timestamp{58:55}
	\section*{\LARGE\S 13 Малая теорема Ферма}
	\label{par:teorFerma}
\end{center}

Проведем эксперимент: посмотрим, насколько интересными свойствами обладают степени натуральных чисел по простому модулю $p$. Будем рассматривать выражения вида $n^p \mod p$.

\begin{example*}
	Пусть $n=3$, $p=5$.
	
	$\begin{cases}
		3^1 \equiv 3 \pmod 5
		\\
		3^2 = 9 \equiv 4 \pmod 5
		\\
		3^3 = 4 \cdot 3 = 12 \equiv 2 \pmod 5
		\\
		3^4 = 2 \cdot 3 = 6 \equiv 1 \pmod 5
		\\
		3^5 = 1 \cdot 3 = 3 \equiv 3 \pmod 5
	\end{cases}$
	\label{exp:sravn1}
\end{example*}

\noindent Мы можем предположить, что $n^p \equiv n \pmod p$~--- это и есть малая теорема Ферма. Рассмотрим еще один пример.

\begin{example*}
	Пусть $n=4$, $p=5$.
	
	$\begin{cases}
	4^1 \equiv (-1) \equiv 4 \pmod 5
	\\
	4^2 \equiv (-1) \cdot (-1) = 1 \pmod 5
	\\
	4^3 = 1 \cdot 4 \equiv 4 \pmod 5
	\\
	4^4 \equiv (-1) \cdot (-1) = 1 \pmod 5
	\\
	4^5 = 1 \cdot 4 \equiv 4 \pmod 5
	\end{cases}$
	\label{exp:sravn2}
\end{example*}

\begin{note}
	Как видно из примера~\ref{exp:sravn2}, равенство не обязательно наступает только на $p$-ом шаге.
\end{note}

\begin{theorem}[Малая теорема Ферма]
	$n^p \equiv n \pmod p$
\end{theorem}

\begin{proof}
	Рассмотрим полный набор классов вычетов, кроме $\klas{0}$.
	
	$1, 2, \dots, (p-1)$~--- ненулевые остатки. Умножим их на $n$. $(\diamond)$
	
	$n, 2n, \dots, n \cdot (p-1)$~--- дадут ли эти числа те же самые остатки (будет ли это тот же набор классов вычетов)? $(\ast)$
	
	\begin{lemma}
		\timestamp{65:37}
		Если $p$~--- простое, $n$ не делится на $p$, то $(\ast)$~--- все ненулевые остатки от деления на $p$.
	\end{lemma}
	
	\begin{proof}
		От противного: пусть $in$ и $jn$ дают одинаковые остатки ($i \ne j$).
		\[in \equiv jn \pmod p \iff \text{[ по теореме ]} \iff in - jn \divs p \iff  n \cdot (i - j) \divs p \implies \]
		\[ \implies \text{[ т.к. $n$ не делится на $p$ ]} \implies (i-j) \divs p \]
		
		\begin{figure}[H]
			\centering
			\tikz{
				\draw [thick] (-3,0) -- (3,0);
				\draw [thick] (-1.8,0.35) -- (-2,0.35) -- (-2,-0.35) -- (-1.8,-0.35);
				\draw [thick] (1.8,0.35) -- (2,0.35) -- (2,-0.35) -- (1.8,-0.35);
				
				\path [fill=black] (-1,0) circle (0.7mm);
				\path [fill=black] (1,0) circle (0.7mm);
				
				\draw [thick] (-1,0) arc (180:0:1 and 0.25);
				\draw [thick] (-2,0) arc (180:0:2 and 0.6);
				
				\coordinate [label=below:$i$] (A) at (-1,0);
				\coordinate [label=below:$j$] (B) at (1,0);
				\coordinate [label=-135:$0$] (C) at (-2,0);
				\coordinate [label=-45:$p-1$] (D) at (2,0);
			}
			\caption{\small $i$ и $j$ относительно других остатков}
			\label{fig:lemma1}
		\end{figure}
	\noindent $|i-j|$ не превышает $|p-1-0|$ и поэтому строго меньше $p$ ($i$ и $j$~--- остатки от деления на $p$)
	 
	$\begin{cases}
	|i-j|<p \\
	|i-j| \divs p
	\end{cases} \implies i-j=0 \implies i=j$~--- противоречие!!!
		
	\end{proof}
	
	\noindent Таким образом множество остатков при умножении их на $n$ не изменилось, значит, остатки произведений ($\ast$) и ($\diamond$) при делении на $p$ будут одинаковы:
	\[ n \cdot 2n \cdot \ \dots \ \cdot (p-1)n \equiv 1 \cdot 2 \cdot \ \dots \ \cdot (p-1) \pmod p \]
	Вынесем все $n$ из произведения в левой части.
	\[ n^{p-1} \cdot (1 \cdot 2 \cdot \ \dots \ \cdot (p-1)) \equiv 1 \cdot 2 \cdot \ \dots \ \cdot (p-1) \pmod p \]
	Разделим обе части на факториал $(p-1)$.
	\[ n^{p-1} \cdot (1 \cdot 2 \cdot \ \dots \ \cdot (p-1)) \equiv 1 \cdot 2 \cdot \ \dots \ \cdot (p-1) \pmod p \mid / (p-1)! \]
	Домножим на $n$ обе части.
	\[ n^{p-1} \equiv 1 \pmod p \mid \cdot n\]
	\[ n^p \equiv n \pmod p\]

\end{proof}

\begin{note}
	Мы можем разделить на каждый множитель $(p-1)!$ по свойству сравнений, позволяющему делить обе части на множитель, взаимно простой с модулем, а все остатки от $1$ до $p-1$ взаимно просты с $p$.
\end{note}

\begin{center}
	\timestamp{73:45}
	\section*{\LARGE\S 14 Китайская теорема об остатках}
	\label{par:kitteor}
\end{center}

Наименьшее натуральное число, делящееся на $2, 3, 5$~--- $30$.

Наименьшее натуральное число, дающее остаток $1$ при делении на $2, 3, 5$~--- $31$. 

А наименьшее натуральное число, дающее остаток $1$ при делении на $2$, $2$ при делении на $3$, $3$ при делении на $5$, уже так просто не посчитать. На этот вопрос дает ответ китайская теорема об остатках.

\begin{theorem}
	Если:
	
	$\begin{cases}
		x \equiv r_1 \pmod {m_1} \\
		x \equiv r_2 \pmod {m_2} \\
		\vdots \\
		x \equiv r_n \pmod {m_n} \\
	\end{cases} (\forall i, j \ (i \ne j) \ m_i \text{ и } m_j \text{~--- взаимно простые}) \text{, то}$
	
	\[ x \equiv c_1d_1r_1 + c_2d_2r_2 + \ldots + c_nd_nr_n \pmod m, \]где $m=m_1 \cdot m_2 \cdot \ldots \cdot m_n$, $c_i =  \displaystyle {{m}\over{m_i}} = m_1 \cdot \ldots \cdot m_{i-1} \cdot m_{i+1} \cdot \ldots \cdot m_n$ (делить никогда не нужно), а $d_i$~--- такое число, что $c_id_i \equiv 1 \pmod {m_i}$
	
\end{theorem}

\begin{note}
	Если модули не взаимно простые, то система сведется либо к описанной в теореме, либо не будет иметь решений.
	
	Заметим, что для ответа используется принцип "разделяй и властвуй". Слагаемое с номером $i$ обеспечивает выполнение $i$-го уравнения, не влияя на выполнение остальных. Это будет видно из следующего доказательства.
\end{note}

\begin{proof}
	Проверим, что построенное решение удовлетворяет каждому из условий. Рассмотрим $i$-ое условие. 
	\[ x \overset{?}{\equiv} r_i \pmod{m_i} \]
	Все слагаемые, кроме $i$-го будут делиться на $m_i$, так как в каждом будет присутствовать $m_i$, следовательно, $x$ будет сравним с $i$-ым слагаемым, а все остальные слагаемые дадут остаток $0$. 
	\[ x = \underbrace{m_2 \cdot \ldots \cdot m_i \cdot \ldots \cdot m_n \cdot d_1 \cdot r_1}_{\divs m_i} + \ldots \equiv c_id_ir_i \pmod {m_i} \]
	Но $c_i$ и $d_i$ заданы так, что $c_id_i = 1$, следовательно, останется только $r_i$.
	\[ \underbrace{c_id_i}_{\equiv 1}r_i \equiv r_i \pmod {m_i} \]
	
\end{proof}

\begin{remark}[как считать обратное]
	\timestamp{84:30}
	\begin{lemma}
		Если $m$ и $n$~--- взаимно простые, то существует такое $n'$, что $n \cdot n' \equiv 1 \pmod m$.
	\end{lemma}
	
	\begin{proof}
		Доказательство также является алгоритмом нахождения обратного:
		\[
			n \cdot n' \equiv 1 \pmod m \iff \text{[по теореме]} \iff n \cdot n' - 1 \divs m
		\]
		Обозначим $n'$ за $x$. Т.к. $(n \cdot n' - 1)$ делится на $m$, то существует число, обозначим его $y$, такое, что $my = n \cdot n' - 1$, тогда 
		\[
			n \cdot x - 1 = m \cdot y
		\]
		Перенесем $m \cdot y$ в левую часть:
		\[
		 	n \cdot x - m \cdot y = 1 
		\]
		Это диафантово уравнение. Причем при нахождении ответа нас интересует только $x$.
		
	\end{proof}
\end{remark}

\end{document}
