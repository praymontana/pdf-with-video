\documentclass[russian]{lecture-notes}

\usepackage[final]{graphicx}
\usepackage{amsmath}
\usepackage{amssymb}
\usepackage{tikz}
\usepackage{timestamps}

\title{Раскраска графа}
\lecturer{Поздняков Сергей Николаевич}
\notesauthor{Ковалева Ксения}
\date{29.05.2018}

\begin{document}
    \youtubevideo{Aly9V0rZj6I}

    \maketitle
    \section*{Непланарные графы}
    \timestamp{00:00}

    Перед тем, как перейти к раскраске графа, рассмотрим понятие непланарных графов.

    \timestamp{00:17}
    \begin{definition}
        Непланарным графом называется такой граф, который нельзя изобразить на плоскости так, чтобы его ребра не пересекались.
    \end{definition}

    \timestamp{00:13}
    \begin{theorem} [Понтрягина-Куратовского]
        Любой непланарный граф сводится или к полному пятивершинному графу ($K_5$), или к двудольному графу $K_{3,3}$.
    \end{theorem}

    Другими словами, иных непланарных графов не существует, то есть внутри любого другого графа можно найти один из двух графов, представленных ниже:\\

    \begin{figure}
        \centering
        \tikz{
        \draw (0,0) -- (-0.5,1.5) -- (0.9, 2.5) -- (2.4, 1.5) -- (1.9,0) -- cycle;;
        \draw (0,0) -- (0.9, 2.5);
        \draw (0.9, 2.5) -- (1.9,0);
        \draw (-0.5,1.5) -- (1.9,0);
        \draw (-0.5,1.5) -- (2.4, 1.5);
        \draw (0,0) -- (2.4, 1.5);

        \draw [fill=black] (0,0) circle (1 mm);
        \draw [fill=black] (-0.5,1.5) circle (1 mm);
        \draw [fill=black] (0.9, 2.5) circle (1 mm);
        \draw [fill=black] (2.4, 1.5) circle (1 mm);
        \draw [fill=black] (1.9,0) circle (1 mm);

        \coordinate [label=above:$1$] (1) at (-0.5,1.6);
        \coordinate [label=right:$2$] (2) at (1, 2.6);
        \coordinate [label=above:$3$] (3) at (2.4,1.6);
        \coordinate [label=right:$4$] (4) at (1.9,0);
        \coordinate [label=left:$5$] (5) at (-0.1,0);

        \draw (5,0) -- (5,2.5);
        \draw (7,0) -- (7,2.5);
        \draw (9,0) -- (9,2.5);
        \draw (5,0) -- (7,2.5);
        \draw (5,0) -- (9,2.5);
        \draw (7,0) -- (5,2.5);
        \draw (7,0) -- (9,2.5);
        \draw (9,0) -- (5,2.5);
        \draw (9,0) -- (7,2.5);

        \draw [fill=black] (5,0) circle (1 mm);
        \draw [fill=black] (5,2.5) circle (1 mm);
        \draw [fill=black] (7,0) circle (1 mm);
        \draw [fill=black] (7,2.5) circle (1 mm);
        \draw [fill=black] (9,0) circle (1 mm);
        \draw [fill=black] (9,2.5) circle (1 mm);

        \coordinate [label=right:$1$] (1) at (5.1,2.5);
        \coordinate [label=right:$2$] (2) at (7.1,2.5);
        \coordinate [label=right:$3$] (3) at (9.1,2.5);
        \coordinate [label=right:$4$] (4) at (5.1,0);
        \coordinate [label=right:$5$] (5) at (7.1,0);
        \coordinate [label=right:$6$] (6) at (9.1,0);

        \coordinate [label=below:$K_5$] ($K_5$) at (0.95,-0.5);
        \coordinate [label=right:$K_{3,3}$] ($K_{3,3}$) at (6.5,-0.9);
        }
        \caption{Графы $K_5$ и $K_{3,3}$.}\label{fig:k5k33}
    \end{figure}

    \begin{example*}
        \timestamp{01:18}
        Рассмотрим граф Петерсена.

        \begin{figure}
            \centering
            \tikz{
            \draw (0,0) -- (-0.5,1.5) -- (0.9, 2.5) -- (2.4, 1.5) -- (1.9,0) -- cycle;
            \draw (0.25,1.4) -- (1.6,1.4);
            \draw (0.25,1.4) -- (1.35,0.6);
            \draw (1.6,1.4) -- (0.5,0.6);
            \draw (0.9,1.9) -- (1.35,0.6);
            \draw (0.9,1.9) -- (0.5,0.6);

            \draw (0.25,1.4) -- (-0.5,1.5);
            \draw (1.6,1.4) -- (2.4, 1.5);
            \draw (0.9,1.9) -- (0.9, 2.5);
            \draw (0.5,0.6) -- (0,0);
            \draw (1.35,0.6) -- (1.9,0);

            \draw [fill=black] (0,0) circle (1 mm);
            \draw [fill=black] (-0.5,1.5) circle (1 mm);
            \draw [fill=black] (0.9, 2.5) circle (1 mm);
            \draw [fill=black] (2.4, 1.5) circle (1 mm);
            \draw [fill=black] (1.9,0) circle (1 mm);
            \draw [fill=black] (0.25,1.4) circle (1 mm);
            \draw [fill=black] (1.6,1.4) circle (1 mm);
            \draw [fill=black] (0.9,1.9) circle (1 mm);
            \draw [fill=black] (0.5,0.6) circle (1 mm);
            \draw [fill=black] (1.35,0.6) circle (1 mm);
            }
            \caption{Граф Петерсена.}\label{fig:peterson1}
        \end{figure}

        Возникает вопрос: является ли этот граф непланарным? Согласно теореме Понтрягина—Куратовского нужно {\em выбрать подграф} (иными словами, "выбросить" некоторые ребра графа), а затем, если найдется ребро, на котором есть вершины степени 2, заменить его на ребро с вершинами степени 1 (рис.~\ref{fig:graphs}).

        \begin{figure}
            \centering
            \tikz{
            \draw (-0.5,1.5) -- (2.4,1.5);
            \draw (-0.5,0.5) -- (2.4,0.5);
            \draw [->] (1,1.3) -- (1,0.7);

            \draw [fill=black] (-0.5,1.5) circle (1 mm);
            \draw [fill=black] (2.4,1.5) circle (1 mm);
            \draw [fill=black] (1, 1.5) circle (1 mm);
            \draw [fill=black] (-0.5,0.5) circle (1 mm);
            \draw [fill=black] (2.4,0.5) circle (1 mm);
            }
            \caption{Наглядная иллюстрация замены рёбер степени 2.}\label{fig:graphs}
        \end{figure}

        Выберем в этом графе, например, подграф вида $K_{3,3}$: для этого нужно связать одну его вершину с тремя другими. Обозначим вершины следующим образом:

        \begin{figure}
            \centering
            \tikz{
            \draw (5,0) -- (5,2.5);
            \draw (7,0) -- (7,2.5);
            \draw (9,0) -- (9,2.5);
            \draw (5,0) -- (7,2.5);
            \draw (5,0) -- (9,2.5);
            \draw (7,0) -- (5,2.5);
            \draw (7,0) -- (9,2.5);
            \draw (9,0) -- (5,2.5);
            \draw (9,0) -- (7,2.5);

            \draw [fill=black] (5,0) circle (1 mm);
            \draw [fill=black] (5,2.5) circle (1 mm);
            \draw [fill=black] (7,0) circle (1 mm);
            \draw [fill=black] (7,2.5) circle (1 mm);
            \draw [fill=black] (9,0) circle (1 mm);
            \draw [fill=black] (9,2.5) circle (1 mm);

            \coordinate [label=right:$x_1$] ($x_1$) at (5.1,2.5);
            \coordinate [label=right:$x_2$] ($x_2$) at (7.1,2.5);
            \coordinate [label=right:$x_3$] ($x_3$) at (9.1,2.5);
            \coordinate [label=right:$y_1$] ($y_1$) at (5.1,0);
            \coordinate [label=right:$y_2$] ($y_2$) at (7.1,0);
            \coordinate [label=right:$y_3$] ($y_3$) at (9.1,0);
            }
            \caption{Обозначение вершин графа $K_{3,3}$.}\label{fig:choose1}
        \end{figure}
        Тогда мы можем обозначить вершины графа Петерсена так же, пользуясь заменами рёбер и ненужных вершин:

        \begin{figure}
            \centering
            \tikz{
            \draw (0,0) -- (-0.5,1.5) -- (0.9, 2.5) -- (2.4, 1.5) -- (1.9,0) -- cycle;
            \draw (0.25,1.4) -- (1.6,1.4);
            \draw (0.25,1.4) -- (1.35,0.6);
            \draw (1.6,1.4) -- (0.5,0.6);
            \draw (0.9,1.9) -- (1.35,0.6);
            \draw (0.9,1.9) -- (0.5,0.6);

            \draw (0.25,1.4) -- (-0.5,1.5);
            \draw (1.6,1.4) -- (2.4, 1.5);
            \draw (0.9,1.9) -- (0.9, 2.5);
            \draw (0.5,0.6) -- (0,0);
            \draw (1.35,0.6) -- (1.9,0);

            \draw [fill=black] (0,0) circle (1 mm);
            \draw [fill=black] (-0.5,1.5) circle (1 mm);
            \draw [fill=black] (0.9, 2.5) circle (1 mm);
            \draw [fill=black] (2.4, 1.5) circle (1 mm);
            \draw [fill=black] (1.9,0) circle (1 mm);
            \draw [fill=black] (0.25,1.4) circle (1 mm);
            \draw [fill=black] (1.6,1.4) circle (1 mm);
            \draw [fill=black] (0.9,1.9) circle (1 mm);
            \draw [fill=black] (0.5,0.6) circle (1 mm);
            \draw [fill=black] (1.35,0.6) circle (1 mm);

            \coordinate [label=right:$x_1$] ($x_1$) at (0.9,2);
            \coordinate [label=below:$x_2$] ($x_2$) at (0,-0.1);
            \coordinate [label=below:$x_3$] ($x_3$) at (1.7,1.4);
            \coordinate [label=above:$y_1$] ($y_1$) at (1, 2.5);
            \coordinate [label=right:$y_2$] ($y_2$) at (0.6,0.6);
            \coordinate [label=right:$y_3$] ($y_3$) at (1.36,0.6);

            \draw [->] (3,1.4) -- (4,1.4);

            \draw (5,0) -- (4.5,1.5) -- (5.9, 2.5) -- (7.4, 1.5);
            \draw [dashed] (7.4, 1.5) -- (6.9,0);
            \draw (6.9, 0) -- (5,0);
            \draw (5.25,1.4) -- (6.6,1.4);
            \draw (5.25,1.4) -- (6.35,0.6);
            \draw (6.6,1.4) -- (5.5,0.6);
            \draw (5.9,1.9) -- (6.35,0.6);
            \draw (5.9,1.9) -- (5.5,0.6);

            \draw [dashed] (5.25,1.4) -- (4.5,1.5);
            \draw (6.6,1.4) -- (7.4, 1.5);
            \draw (5.9,1.9) -- (5.9, 2.5);
            \draw (5.5,0.6) -- (5,0);
            \draw (6.35,0.6) -- (6.9,0);

            \draw [fill=black] (5,0) circle (1 mm);
            \draw [fill=white] (4.5,1.5) circle (1 mm);
            \draw [fill=black] (5.9, 2.5) circle (1 mm);
            \draw [fill=white] (7.4, 1.5) circle (1 mm);
            \draw [fill=white] (6.9,0) circle (1 mm);
            \draw [fill=white] (5.25,1.4) circle (1 mm);
            \draw [fill=black] (6.6,1.4) circle (1 mm);
            \draw [fill=black] (5.9,1.9) circle (1 mm);
            \draw [fill=black] (5.5,0.6) circle (1 mm);
            \draw [fill=black] (6.35,0.6) circle (1 mm);

            \coordinate [label=right:$x_1$] ($x_1$) at (5.9,2);
            \coordinate [label=below:$x_2$] ($x_2$) at (5,-0.1);
            \coordinate [label=below:$x_3$] ($x_3$) at (6.7,1.4);
            \coordinate [label=above:$y_1$] ($y_1$) at (6, 2.5);
            \coordinate [label=right:$y_2$] ($y_2$) at (5.6,0.6);
            \coordinate [label=right:$y_3$] ($y_3$) at (6.36,0.6);
            }
            \caption{Преобразование вершин графа Петерсена в соответствии с рис.~\ref{fig:choose1}. Пунктирные рёбра и незакрашенные вершины удаляются.}\label{fig:choose2}
        \end{figure}
        Возникает вопрос: почему мы расположили обозначения именно так? Нам нужно выбрать такой подграф, чтобы каждая вершина $x_{i}$ ($i$ = 1, 2, 3) была связана с тремя $y_{j}$ ($j$ = 1, 2, 3). Данный граф позволяет нам это сделать, следовательно, граф Петерсена является непланарным.

    \end{example*}

    \timestamp{06:12}

    \begin{example*}
        Рассмотрим второй способ проверки графа Петерсена на непланарность~--- теорему Вагнера.
        Эта теорема основывается на принципе {\em стягивания рёбер}, то есть замены ребра на одну вершину:

        \begin{figure}
            \centering
            \tikz{
            \draw (1,1.5) -- (2,1.5);
            \draw (0,2) -- (1,1.5);
            \draw (0,1) -- (1,1.5);
            \draw (3,2.2) -- (2,1.5);
            \draw (3,1.6) -- (2,1.5);
            \draw (3,0.7) -- (2,1.5);

            \draw (1.5,0) -- (0.5,0.5);
            \draw (1.5,0) -- (0.5,-0.5);
            \draw (1.5,0) -- (2.5,0.7);
            \draw (1.5,0) -- (2.5,0.1);
            \draw (1.5,0) -- (2.5,-0.8);
            \draw [->] (1.5,1.3) -- (1.5,0.5);

            \draw [fill=black] (1,1.5) circle (1 mm);
            \draw [fill=black] (2,1.5) circle (1 mm);
            \draw [fill=black] (1.5,0) circle (1 mm);
            }
            \caption{Наглядная иллюстрация стягивания рёбер.}\label{fig:OKwagner}
        \end{figure}
        Применим данный способ к нашему графу:

        \begin{figure}
            \centering
            \tikz{
            \draw (0,0) -- (-0.5,1.5) -- (0.9, 2.5) -- (2.4, 1.5) -- (1.9,0) -- cycle;
            \draw (0.25,1.4) -- (1.6,1.4);
            \draw (0.25,1.4) -- (1.35,0.6);
            \draw (1.6,1.4) -- (0.5,0.6);
            \draw (0.9,1.9) -- (1.35,0.6);
            \draw (0.9,1.9) -- (0.5,0.6);

            \draw (0.25,1.4) -- (-0.5,1.5);
            \draw (1.6,1.4) -- (2.4, 1.5);
            \draw (0.9,1.9) -- (0.9, 2.5);
            \draw (0.5,0.6) -- (0,0);
            \draw (1.35,0.6) -- (1.9,0);

            \draw [fill=black] (0,0) circle (1 mm);
            \draw [fill=black] (-0.5,1.5) circle (1 mm);
            \draw [fill=black] (0.9, 2.5) circle (1 mm);
            \draw [fill=black] (2.4, 1.5) circle (1 mm);
            \draw [fill=black] (1.9,0) circle (1 mm);
            \draw [fill=black] (0.25,1.4) circle (1 mm);
            \draw [fill=black] (1.6,1.4) circle (1 mm);
            \draw [fill=black] (0.9,1.9) circle (1 mm);
            \draw [fill=black] (0.5,0.6) circle (1 mm);
            \draw [fill=black] (1.35,0.6) circle (1 mm);

            \draw [->] (3,1.4) -- (4,1.4);

            \draw (5,0) -- (4.5,1.5) -- (5.9, 2.5) -- (7.4, 1.5) -- (6.9,0) -- cycle;;
            \draw (5,0) -- (5.9, 2.5);
            \draw (5.9, 2.5) -- (6.9,0);
            \draw (4.5,1.5) -- (6.9,0);
            \draw (4.5,1.5) -- (7.4, 1.5);
            \draw (5,0) -- (7.4, 1.5);

            \draw [fill=black] (5,0) circle (1 mm);
            \draw [fill=black] (4.5,1.5) circle (1 mm);
            \draw [fill=black] (5.9, 2.5) circle (1 mm);
            \draw [fill=black] (7.4, 1.5) circle (1 mm);
            \draw [fill=black] (6.9,0) circle (1 mm);
            }
            \caption{Преобразование вершин графа Петерсена в соответствии с рис.~\ref{fig:OKwagner}.}\label{fig:RESwagner}
        \end{figure}

        Данным способом мы привели граф Петерсена к графу вида $K_5$, что доказывает его непланарность.


    \end{example*}
    В качестве итога приведем формулировку:

    \begin{theorem} [Понтрягина—Куратовского.] Граф является непланарным тогда и только тогда, когда в нём существует подграф, {\em гомеоморфный} $K_5$ или $K_{3,3}$.
    \end{theorem}

    \begin{theorem}[Вагнера] Требуется сформулировать самостоятельно.
    \end{theorem}

    \begin{remark}
        Когда речь идет о гомеоморфизмах, имеются ввиду непрерывные преобразования.
    \end{remark}

    Например, буквы "С" и "Г" гомеоморфны, так как, согнув букву "Г", мы можем получить букву "С". В то же время буква "Т" и буква "С" не гомеоморфны, но гомеоморфны буквы "Т" и "Е".

    Если две геометрические фигуры могут быть непрерывными преобразованиями переведены друг в друга, они называются {\em гомеоморфными}.

    \section*{Замечание о транзитивности замыкания}
    \timestamp{10:52}
    В ходе предыдущей лекции было сформулировано два алгоритма: {\em перемножение матриц} и {\em алгоритм Уоршалла} и был поднят вопрос, существует ли более эффективный алгоритм. Методом перемножения матриц была достигнута трудоемкость $N^4$, а алгоритмом Уоршалла~--- $N^3$.

    Помимо этих двух алгоритмов, существует еще один, связанный с перемножением матриц~--- алгоритм Штрассена. Он основан на следующей идее: если мы перемножаем матрицы $2\times2$, то на каждый элемент матрицы приходится два умножения, следовательно, на всю матрицу будет 8 умножений.

    \[
        \begin{pmatrix}
            a_1 & b_1 \\
            c_1 & d_1
        \end{pmatrix}   *
        \begin{pmatrix}
            a_2 & b_2 \\
            c_2 & d_2
        \end{pmatrix} =
        \begin{pmatrix}
            a_3=a_1a_2+b_1c_2 & b_3=a_1b_2+b_1d_2 \\
            c_3=c_1a_2+d_1c_2 & d_3=c_1b_2+d_1d_2
        \end{pmatrix}.
    \]

    Идея: можно ли преобразовать формулу умножения так, чтобы на всю матрицу вышло 7 умножений?

    Эта идея переносится на произвольные матрицы так: матрица состоит из малых (в нашем случае, матриц $2\times2$), и ее перемножение выглядит таким же образом. Результатом такой работы является трудоемкость, примерно равная $N^{2.8\ldots}$. Также имеются работы, где трудоемкость достигает $N^{2.3\ldots}$.

    Кроме того, когда мы писали общий цикл, то производили умножение матрицы на себя. Но мы можем применить алгоритм быстрого возведения в степень, тогда работа будет проводиться с логарифмом, а трудоемкость примерно равна $\log N^{2.3\ldots}$.

    Однако данный алгоритм эффективен только при работе с большими объемами, поэтому предыдущие два алгоритма используются чаще.

    \section*{\S9. Раскраска графов}
    \timestamp{18:39}

    \normalsize
    \begin{definition}
        Граф называется {\em раскрашенным}, если каждой его вершине присвоен некоторый цвет, который задается целым числом так, что если вершины соединены одним ребром, то они раскрашены в разные цвета (такая раскраска вершин называется {\em правильной}).
    \end{definition}

    \begin{example*}
        Минимальное количество красок, которыми можно правильно раскрасить граф, изображенный на рис.~\ref{fig:graph}, равно двум: здесь можно провести аналогию с шахматной доской.

        \begin{figure}
            \centering
            \tikz{
            \draw [step=1 cm] (0,0) grid (4,4);

            \draw [fill=blue] (0,0) circle (1 mm);
            \draw [fill=blue] (2,0) circle (1 mm);
            \draw [fill=blue] (4,0) circle (1 mm);
            \draw [fill=blue] (0,2) circle (1 mm);
            \draw [fill=blue] (0,4) circle (1 mm);
            \draw [fill=blue] (1,1) circle (1 mm);
            \draw [fill=blue] (1,3) circle (1 mm);
            \draw [fill=blue] (2,2) circle (1 mm);
            \draw [fill=blue] (2,4) circle (1 mm);
            \draw [fill=blue] (3,1) circle (1 mm);
            \draw [fill=blue] (3,3) circle (1 mm);
            \draw [fill=blue] (4,2) circle (1 mm);
            \draw [fill=blue] (4,4) circle (1 mm);

            \draw [fill=red] (0,1) circle (1 mm);
            \draw [fill=red] (0,3) circle (1 mm);
            \draw [fill=red] (1,0) circle (1 mm);
            \draw [fill=red] (1,2) circle (1 mm);
            \draw [fill=red] (1,4) circle (1 mm);
            \draw [fill=red] (2,1) circle (1 mm);
            \draw [fill=red] (2,3) circle (1 mm);
            \draw [fill=red] (3,0) circle (1 mm);
            \draw [fill=red] (3,2) circle (1 mm);
            \draw [fill=red] (3,4) circle (1 mm);
            \draw [fill=red] (4,1) circle (1 mm);
            \draw [fill=red] (4,3) circle (1 mm);
            }
            \caption{Правильно раскрашенный граф {\em $ 4\times4 $}.}\label{fig:graph}
        \end{figure}
    \end{example*}
    \timestamp{21:59}
    \begin{example*}[Задача о четырех красках]
        При раскрашивании административных делений географической карты используется такое правило: если две страны имеют общую границу, то они должны быть раскрашены разными красками. Если граница точечная, то они могут быть раскрашены одной краской.

        Какое минимальное количество красок необходимо для того, чтобы  раскрасить карту?

        \begin{proof}[Решение] Приведем классический пример карты для этой задачи (рис.~\ref{fig:map}, a). Данная задача легко сводится к задаче правильной раскраски графа: в каждой области нужно взять по точке и провести ребра графа между этими точками так, чтобы они пересекали границу (рис.~\ref{fig:map}, b). Легко увидеть, что минимальное количество красок, которым можно правильно раскрасить эту карту, равно четырем.

        \begin{figure}
            \centering
            \tikz{
            \draw [fill=green!80!black] (0,0)
            -- (0,1.5) arc (90:210:1.5cm) -- cycle;
            \draw [fill=yellow!50!red] (0,0) -- (-1.3,-0.75) arc (210:330:1.5cm) -- cycle;
            \draw [fill=blue!80!black] (0,0) -- (0,1.5) arc (90:-30:1.5cm) -- cycle;
            \draw [fill=yellow] (0,0) circle (0.75 cm);

            \draw (5,-0.4) -- (5,1.25);
            \draw (5,-0.4) -- (3.5,-1.25);
            \draw (5,-0.4) -- (6.5,-1.25);
            \draw (5,1.25) -- (3.5,-1.25);
            \draw (5,1.25) -- (6.5,-1.25);
            \draw (6.5,-1.25) -- (3.5,-1.25);
            \draw [fill=yellow] (5,-0.4) circle (3 mm);
            \draw [fill=green!80!black] (5,1.25) circle (3 mm);
            \draw [fill=yellow!50!red] (3.5,-1.25) circle (3 mm);
            \draw [fill=blue!80!black] (6.5,-1.25) circle (3 mm);

            \coordinate [label=below:$ a. $] ($ a. $) at (0,-1.6);
            \coordinate [label=right:$ b. $] ($ b. $) at (4.75,-1.8);
            }
            \caption{Карта, которую можно раскрасить только четырьмя цветами}\label{fig:map}
        \end{figure}
        \end{proof}
    \end{example*}

    \begin{conjecture} можно ли любую карту раскрасить четырьмя красками?
    \end{conjecture}
    \timestamp{25:40}
    \begin{example*}[Задача о количестве раскрасок].

    Возьмем граф, изображенный на рис.~\ref{fig:counting}, а. Сколько всего у него существует правильных раскрасок двумя, тремя красками?

    \begin{proof}[Решение]
        Очевидно, раскрасить двумя красками этот граф невозможно. Тремя же красками можно раскрасить его 6-ю способами (три способа раскрасить первую вершину, два -- вторую и один -- первую), или $3!$. Если бы мы не учитывали графы, которые можно получить вращением, ответом было бы число 2, но это другая задача.

        Возьмем граф, изображенный на рис.~\ref{fig:counting}, b, и решим ту же самую задачу. В ранее рассмотренной задаче мы доказали, что такой граф не может быть раскрашен менее, чем четырьмя красками. Таким образом, для любого количества красок меньше четырех ответом будет 0. При раскрашивании четырьмя красками ответ получается способом, аналогичным использованному на полном трехвершинном графе~--- $4!$ или 24.
    \end{proof}

    \begin{figure}
        \centering
        \tikz{
        \draw (0,1.25) -- (-1.5,-1.25);
        \draw (0,1.25) -- (1.5,-1.25);
        \draw (1.5,-1.25) -- (-1.5,-1.25);
        \draw [fill=green!80!black] (0,1.25) circle (3 mm);
        \draw [fill=yellow!50!red] (-1.5,-1.25) circle (3 mm);
        \draw [fill=blue!80!black] (1.5,-1.25) circle (3 mm);

        \draw (5,-0.4) -- (5,1.25);
        \draw (5,-0.4) -- (3.5,-1.25);
        \draw (5,-0.4) -- (6.5,-1.25);
        \draw (5,1.25) -- (3.5,-1.25);
        \draw (5,1.25) -- (6.5,-1.25);
        \draw (6.5,-1.25) -- (3.5,-1.25);
        \draw [fill=yellow] (5,-0.4) circle (3 mm);
        \draw [fill=green!80!black] (5,1.25) circle (3 mm);
        \draw [fill=yellow!50!red] (3.5,-1.25) circle (3 mm);
        \draw [fill=blue!80!black] (6.5,-1.25) circle (3 mm);

        \coordinate [label=below:$ a. $] ($ a. $) at (0,-1.6);
        \coordinate [label=right:$ b. $] ($ b. $) at (4.75,-1.8);
        }
        \caption{Раскрашенные полные графы с 3 и 4 вершинами}
        \label{fig:counting}
    \end{figure}
    Связано это с тем, что в обоих рассмотренных случаях каждая вершина была связана с каждой, а количество красок было равно количеству вершин. В таком случае количество раскрасок $K_n = n!$, где n~--- количество вершин. В случае, если красок было бы больше, чем вершин ($m$ красок), формула выглядела бы так: $m(m-1)(m-2)\ldots(m-n+1)$. Это называется {\em числом размещений}.
    \vspace {0.25 cm}

    Теперь давайте рассмотрим граф, изображенный на рис.~\ref{fig:method}~--- неполный четырехвершинный граф. Рассмотрим его вершины, расположенные по диагонали: либо они раскрашены одинаково, либо по-разному. Если одинаково, то их можно склеить. Если они раскрашены по-разному, то можно провести ребро, которое их соединяет, и это не изменит числа раскрасок. Мы можем идти по этому алгоритму, пока не дойдем до полного графа, а как считать количество раскрасок полного графа, мы уже знаем.

    \begin{figure}
        \centering
        \tikz{
        \draw (-1,1) -- (1,1);
        \draw (-1,1) -- (-1,-1);
        \draw (-1,-1) -- (1,-1);
        \draw (1,-1) -- (1,1);

        \draw [->, thick] (-1.2, -1.2) -- (-1.5, -1.5);
        \draw [->, thick] (1.2, -1.2) -- (1.5, -1.5);

        \draw (-1.7, -1.7) -- (-1.7, -3.7);
        \draw (-1.7, -3.7) -- (-3.7, -3.7);
        \draw (1.7,-1.7) -- (3.7,-1.7);
        \draw (3.7,-1.7) -- (3.7,-3.7);
        \draw (3.7,-3.7) -- (1.7,-3.7);
        \draw (1.7,-3.7) -- (1.7,-1.7);
        \draw (1.7,-1.7) -- (3.7,-3.7);

        \draw [->, thick] (-3.9, -3.9) -- (-4.1, -4.1);
        \draw [->, thick] (-1.9, -3.9) -- (-1.7, -4.1);
        \draw [->, thick] (3.9, -3.9) -- (4.1, -4.1);
        \draw [->, thick] (1.9, -3.9) -- (1.7, -4.1);

        \draw (-4.3, -4.3) -- (-4.3, -6.3);
        \draw (-1.2, -4.3) -- (-1.2, -6.3);
        \draw (-1.2, -6.3) -- (-3.2, -6.3);
        \draw (-3.2, -6.3) -- (-1.2, -4.3);
        \draw (1.2, -4.3) -- (1.2, -6.3);
        \draw (1.2, -6.3) -- (3.2, -6.3);
        \draw (1.2, -4.3) -- (3.2, -6.3);
        \draw (3.9,-4.3) -- (5.9,-4.3);
        \draw (5.9,-4.3) -- (5.9,-6.3);
        \draw (5.9,-6.3) -- (3.9,-6.3);
        \draw (3.9,-6.3) -- (3.9,-4.3);
        \draw (3.9,-4.3) -- (5.9,-6.3);
        \draw (3.9,-6.3) -- (5.9,-4.3);

        \draw [fill=white] (-1,1) circle (2 mm);
        \draw [fill=white] (-1,-1) circle (2 mm);
        \draw [fill=white] (1,-1) circle (2 mm);
        \draw [fill=white] (1,1) circle (2 mm);
        \draw [fill=white] (-1.7,-1.7) circle (2 mm);
        \draw [fill=white] (-3.7,-3.7) circle (2 mm);
        \draw [fill=white] (1.7,-3.7) circle (2 mm);
        \draw [fill=white] (3.7,-1.7) circle (2 mm);
        \draw [fill=blue] (-1.7,-3.7) circle (2 mm);
        \draw [fill=blue] (3.7,-3.7) circle (2 mm);
        \draw [fill=blue] (-4.3, -6.3) circle (2 mm);
        \draw [fill=blue] (-1.2, -6.3) circle (2 mm);
        \draw [fill=blue] (3.2, -6.3) circle (2 mm);
        \draw [fill=blue] (5.9,-6.3) circle (2 mm);
        \draw [fill=orange] (1.7,-1.7) circle (2 mm);
        \draw [fill=orange] (1.2, -4.3) circle (2 mm);
        \draw [fill=orange] (3.9,-4.3) circle (2 mm);
        \draw [fill=green] (-4.3, -4.3) circle (2 mm);
        \draw [fill=green] (-1.2, -4.3) circle (2 mm);
        \draw [fill=green] (5.9, -4.3) circle (2 mm);
        \draw [fill=yellow] (-3.2, -6.3) circle (2 mm);
        \draw [fill=yellow] (1.2, -6.3) circle (2 mm);
        \draw [fill=yellow] (3.9, -6.3) circle (2 mm);
        }
        \caption{Получение количества возможных раскрасок графа}\label{fig:method}
    \end{figure}

    Посчитав количество раскрасок для каждого случая, мы получим $12+24+24+24 = 84$, что и является ответом данной задачи.

    \end{example*}

    \timestamp{32:39}

    {\em Замечание о хроматическом многочлене}

    Если вместо 4-х цветов у нас есть $x$ цветов, то для удобства решения предыдущей задачи можно составить {\em хроматический многочлен}. Он будет иметь вид: $\chi(x) = x(x-1)(x-2)(x-3)+2x(x-1)(x-2)+x(x-1)$.

\end{document}
